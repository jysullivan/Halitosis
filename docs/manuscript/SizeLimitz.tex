%
%  SizeLimitz
%
%  Created by Martell on 2012-09-05.
%  Copyright (c) 2012 UBC Fisheries Centre. All rights reserved.
\documentclass[12pt]{article}
%\documentclass[leqno,author,preprint]{nrc2}

% %% Journal-specific information for opening page -- pp.9-11 of guide:
% %% a. numbers:
% \setcounter{page}{1} %% replace 1 with starting page no.
% \volyear{XX}{2012} %% volume, year of journal
% \journal{Can. J. Fish. Aquat. Sci.} %% jrnl. abbrev. (see App.A of guide)
% \journalcode{cjfas} %% jrnl. acro (see App.A of guide)
% \filenumber{} %% NRC file number
% %% \filenumber*{} %% prefixes \filenumber to all page nos.
% %% NOTE: COMMENT OUT class options
% %% pagnf
% %% proof
% %% preprint
% %% trimmarks
% %% once no longer needed
% 
% %% b. dates:
% \received{} %% insert date, no period
% \revreceived{} %% <same>
% \accepted{} %% <same>
% \revaccepted{} %% <same>
% %% \IDdates{} %% <same>. Use for ëRevised ...í etc.
% %% \webpub{} %% insert date
% %% \commdate{} %% <same>
% %% c. miscellaneous:
% %% \assoced{} %% insert name of Associate ed.
% %% \corred{} %% insert name of Corresponding ed.
% %% \dedication{} %% insert text as neede
% %% \abbreviations{} %% insert as needed


% Use utf-8 encoding for foreign characters
\usepackage[utf8]{inputenc}

% Setup for fullpage use
\usepackage{fullpage}

% Bibliography
\usepackage[round]{natbib}	

% Uncomment some of the following if you use the features
%
% Running Headers and footers
%\usepackage{fancyhdr}

% Multipart figures
%\usepackage{subfigure}

% More symbols
\usepackage{amsmath}
\usepackage{amssymb}
\usepackage{latexsym}

% Surround parts of graphics with box
\usepackage{boxedminipage}

% Package for including code in the document
\usepackage{listings}

% If you want to generate a toc for each chapter (use with book)
\usepackage{minitoc}

% This is now the recommended way for checking for PDFLaTeX:
\usepackage{ifpdf}

%\newif\ifpdf
%\ifx\pdfoutput\undefined
%\pdffalse % we are not running PDFLaTeX
%\else
%\pdfoutput=1 % we are running PDFLaTeX
%\pdftrue
%\fi

\ifpdf
\usepackage[pdftex]{graphicx}
\else
\usepackage{graphicx}
\fi
\title{Impacts of size limits, release mortality, growth variation, and cumulative fishing mortality on the harvest policy for Pacific halibut}
\author{ Steven Martell, Ian Stewart, Ray Webster, and Bruce Leaman }
%\address{International Pacific Halibut Commission}

\date{2012-09-05}

\begin{document}

\ifpdf
\DeclareGraphicsExtensions{.pdf, .jpg, .tif}
\else
\DeclareGraphicsExtensions{.eps, .jpg}
\fi


\maketitle

\begin{abstract}
	Harvesting fish is inherently a size-selective process where faster growing individuals recruit to fishing gear at younger ages.  As a consequence of size-selective fishing, faster growing individuals in a given cohort are subject to higher fishing mortality rates than slower growing individuals.  This results in a reduction in the mean size-at-age relative to an unfished population that exhibits the same growth rate.  Limits on the size of fish that can be retained in the fishery can protect spawning biomass, if and only if, release mortality rates are very low.  Release mortality is the fraction of individuals that die after capture and release.  If the release mortality rates are substantially greater than 0, limiting the legal size range could have a negative impact on future yield and spawning biomass if fishing mortality rates are sufficiently high.   Cumulative effects of size-selective fishing are amplified if release mortality rates are high and the minimum size limit is much greater than the size-at-entry to the fishery.  In addition to discards from the directed fishery, accounting for discards associated with non-directed fisheries also lowers estimates of optimal fishing mortality rates.  If bycatch mortality does not scale with overall population size (i.e., a constant amount of bycatch each year), then optimal harvest rates in the directed fishery must be reduced further to account for depensatory mortality rates in non-directed fisheries. 
	
\end{abstract}

%!TEX root = /Users/stevenmartell1/Documents/IPHC/SizeLimitz/docs/manuscript/SizeLimitz.tex
\section{Notes} % (fold)
\label{sec:notes}

\begin{itemize}
	\item Run growth model and calculate F0.1  using only parametric uncertainty in growth.
	\item Calculate F0.1 using parametric uncertainty in growth and a prior distribution for natural mortality.
\end{itemize}


% section notes (end)

\section*{Introduction} % (fold)
\label{sec:introduction}

The current harvest strategy for Pacific halibut includes the use of minimum size limits along with a constant exploitation rate policy to determine annual total catches for commercial and non-commercial fisheries.  Previous studies have examined the impacts of alternative size limits on the yield per recruit and spawning biomass per recruit using steady-state or equilibrium models \citep{clark1995re}, and more recently using a simulation modelling approach \citep{clark2006assessment}.  The current harvest policy for Pacific halibut utilizes a minimum legal size limit of 81.3cm (or 32 inches) total length.  Previous studies have not considered the impacts of size-selective fishing and release mortality rates on optimal harvest policies for Pacific halibut.  Using a minimum size limit, faster growing individuals from a given cohort would experience higher total mortality rates over their lifetime as they recruit to the legal size at a younger age in comparison to slower growing individuals. The cumulative effect over time imposes a higher total mortality rate on faster growing halibut and surviving individuals in the population are biased towards slower growing individuals \citep{Taylor2005}. The cumulative effects of size-selective fishing can give the appearance of declining mean weight-at-age in the population.

Previous yield per recruit and size limit interactions for Pacific halibut did not consider the effects of discard mortality rates. The underlying assumption in this case is that all fish captured and released will survive. \cite{coggins2007ecm} and \cite{pineiii2008car} demonstrated that for a given size limit, increases discard mortality can lower the realized yield per recruit. Moreover, non-zero discard mortality rates would reduce the potential benefit, if any, of an upper size limit as it would afford less opportunity to survive to the upper size limit, and reduce survivorship of captured individuals that not of legal size.

Maximizing the yield per recruit, or maximizing the total equilibrium yield is not always desirable from an economic perspective.  In the case of Pacific halibut, there is a differential price structure for the size of fish landed, where larger fish command much higher per pound prices than smaller fish. In such case, it may be more desirable to fish at rates lower than $\rm{F_{msy}}$, or F$_{0.1}$ in order to maximize the economic value.

In this paper, I develop an equilibrium model that address both the cumulative effects of size-selective fishing and discard mortality rates for Pacific halibut.  Yield per recruit, spawning biomass per recruit, equilibrium yield, depletion and landed value per recruit are compared using five alternative harvest policy scenarios: (1) the current size size limit of 81.3 cm (or 32 inches), (2) no size limits, (3) a 60 cm (or 26 inch) size limit, and (4) a slot limit with lower and upper bounds of 81.3--150 cm, and (5) a slot limit with lower and upper bounds of 60--150 cm.  Each of these five harvest policy scenarios are examined under alternative hypotheses about changes in size-at-age and changes in size-selectivity in the commercial fishery.




% section introduction (end)

%!TEX root = /Users/stevenmartell1/Documents/IPHC/SizeLimitz/docs/manuscript/SizeLimitz.tex
\section*{Methods} % (fold)
\label{sec:methods}

\subsection*{Equilibrium model} % (fold)
\label{sub:equilibrium_model}

For ease of describing the age-structure equilibrium model I ignore the cumulative effects of size-selective fishing and size limits. In an equilibrium setting the annual yield is actually a sum over all ages of the fraction of individuals that die due to fishing multiplied by the total number or biomass of individuals available for harvest.  Thus the equilibrium yield equation can be written as:
\begin{equation}\label{eq:Y_e}
	Y_e = \sum_{a=1}^\infty \frac{B_a F_a [1-\exp(-M_a-F_a)]}{M+F_a}
\end{equation}
where $F_a$ is the age-specific fishing mortality rate which can be parsed as $F_e v_a$, where $v_a$ is the age-specific fraction that is vulnerable to fishing mortality (also termed selectivity in models that do not distinguish between landed and discarded fish).  Biomass at age ($B_a$) is defined as the numbers-at-age ($N_a$) times the average weight-at-age ($w_a$).  Assuming steady-state conditions this can be expressed as the product of recruitment, survivorship and average weight-at-age.  Assuming unfished conditions (i.e., $F_e=0$), survivorship to a given age is given as:
\begin{equation}\label{eq:unfished_survivorship}
	l_a =\begin{cases} 1, & a=1 \\ l_{a-1} \exp(-M_a), &a>1\end{cases}, 
\end{equation}
and survivorship under fished conditions is given by:
\begin{equation}\label{eq:fished_survivorship}
	\acute{l}_a =\begin{cases} 1, & a=1 \\ \acute{l}_{a-1} \exp(-M_a-F_e v_a), &a>1\end{cases}, 
\end{equation}
The age-specific biomass is given as:
\begin{equation} \label{eq:B_a}
	B_a = R_o l_a w_a.
\end{equation}

Substituting \eqref{eq:B_a} into  \eqref{eq:Y_e} and parsing the fishing mortality rate components yields the following expression
\begin{equation}\label{eq:Y_e2}
	Y_e = F_e R_e \sum_{a=1}^\infty \frac{l_a,w_a v_a [1-\exp(-M_a-F_e v_a)]}{M+F_e v_a},
\end{equation}
where $R_e$ is the equilibrium recruitment obtained under a fishing mortality rate $F_e$. The summation term in \eqref{eq:Y_e2} represents the yield per recruit ($\phi_q$), and the yield equation simplifies to:
\begin{equation}\label{eq:Y_e3}
	Y_e = F_e R_e \phi_q.
\end{equation}

For a given equilibrium fishing mortality rate $F_e$, the equilibrium recruitment is a function of the available spawning biomass relative to the unfished spawning biomass. For the Beverton-Holt model, this can be expressed as
\begin{equation}\label{eq:R_e}
	R_e = \frac{R_o (\kappa-\phi_e/\phi_f)}{\kappa -1} 
\end{equation}
where the spawning biomass per recruit $\phi_e$ and $\phi_f$ for unfished and fished conditions, respectively, is based on the survivorship and mature female weight-at-age,  or fecundity-at-age ($f_a$).  Two leading parameters are the unfished age-1 recruits $R_o$, which serves the purpose of providing the scale of the population, and the recruitment compensation parameter $\kappa$ which is defined as the relative improvement in juvenile survival rate as the spawning biomass tends to zero (for the Beverton-Holt model this can be derived from steepness as $\kappa= 4h/[1-h]$).  Spawning biomass per recruit is given by:
\begin{eqnarray}
	\phi_e = \sum_{a=1}^\infty l_a f_a\label{eq:phi_e}\\
	\phi_f = \sum_{a=1}^\infty \acute{l}_a f_a\label{eq:phi_f}
\end{eqnarray}
Note that it is not necessary to have absolute estimates of fecundity and the units cancel out in the $\phi_e/\phi_f$ ration in \eqref{eq:R_e}. What is important is the relative egg contribution by age, and here it is assumed that fecundity is proportional to mature female body weight.

Based on equations \ref{eq:Y_e}--\ref{eq:phi_f} it is now possible to calculate the equilibrium yield given estimates of the following parameters: $\Theta = \{R_o, \kappa, M_a, f_a, w_a, v_a\}$.
% subsection equilibrium_model (end)

\subsection*{Including release mortality} % (fold)
\label{sub:including_release_mortality}
The equilibrium model described in the previous section only considers the case in which all fish captured for a unit of fishing mortality $F_e$ are removed from the population and does account for cases in which some fish captured will be discarded because they are not within the legal size range.  To include the effects of post release mortality associated with size limits, the vulnerability age-schedule ($v_a$) has to be modelled as as a joint probability where there the probability of dying due to fishing is based on the probability of capture and being retained times the probability of being captured, released, and dying after release.  This simple joint probability is as follows:
\begin{equation} \label{eq:v_a}
	v_a = v_c[v_r + (1-v_r)\lambda]
\end{equation}
where $v_a$ is the age-specific mortality rate associated with a unit of fishing mortality, $v_c$ is the age-specific probability of being captured by fishing gear, $v_r$ and $(1-v_r)$ is the age specific retention (release) probability, and $\lambda$ is the probability of dying after being discarded.

To implement the effects of size limits and post-release mortality rates on the equilibrium yield calculations defined in the previous section, we simply substitute \eqref{eq:v_a} for all the $v_a$ terms in equations \ref{eq:fished_survivorship} and \ref{eq:Y_e3} above.
% subsection including_release_mortality (end)

\subsection*{Cumulative effects of size-selective fishing} % (fold)
\label{sub:cumulative_effects_of_size_selective_fishing}
To account for variation in growth and represent the cumulative effects of size-selective fishing, the population is divided into a number of distinct groups ($G$) that each have a unique asymptotic length ($l_{\infty,g}$).  In an unfished population, the assumption is that the distribution of asymptotic lengths is normally distributed with a coefficient of variation of 0.1.  The proportion of recruitment to each of these $G$ groups is assumed to be normally distributed with 99.7\% of all individuals falling within 3 standard deviations of the mean asymptotic length.  There are no assumptions about the composition of the spawning stock biomass and recruitment into each of these groups (i.e., no genetic selection effects due to fishing is assumed), and irrespective of spawning stock size, recruitment to each of these groups follows the same normal distribution.

The per recruit functions described in the previous equations are then modified to include both the age and size effect.  For example the spawning biomass per recruit described in \eqref{eq:phi_f} is now calculated as:
\begin{equation}
	\phi_{f} =\sum_{g=1}^G p_g \sum_{a=1}^\infty \acute{l}_a f_a\label{eq:phi_fg}
\end{equation}
where $p_g \sim N(0,1)$ and computed over 11 discrete intervals from -3 to 3.  In other words, the equilibrium population consists of 11 discrete sub-populations that differ only in their asymptotic lengths and the relative abundance of each sub-population follows a normal distribution.
% subsection cumulative_effects_of_size_selective_fishing (end)

\subsection*{Life-history parameters and price information} % (fold)
\label{sub:life_history_parameters_and_price_information}
For the following analysis, the assumed life-history parameters are listed in Table 1

\begin{table}
	\caption{Life-history parameters used in the equilibrium model.}
	\label{table:Life_history_pars}
	\begin{center}
	\begin{tabular}{lcccc}
		\hline
		 &  & & \multicolumn{2}{c}{Sex-specific}\\
		Parameter         & Symbol     & Value & Female & Male \\
		\hline
		Natural mortality & $M$        &       &   0.15 & 0.18 \\
		Asymptotic length & $l_\infty$ &       &   145  & 110  \\
		Metabolic rate    & $k$        &       &   0.10 & 0.12 \\
		\hline
		
		\multicolumn{3}{l}{Selectivity coefficients}
		 Size (cm) & Female & Male\\
		\hline
		&&  50 & 0.0000& 0.0000\\
		&&  60 & 0.0000& 0.0040\\
		&&  70 & 0.0603& 0.0567\\
		&&  80 & 0.3009& 0.2814\\
		&&  90 & 0.6303& 0.5855\\
		&& 100 & 0.9139& 0.8356\\
		&& 110 & 1.0000& 1.0000\\
		&& 120 & 1.0000& 1.0000\\
		\hline
	\end{tabular}
	\end{center}
\end{table}


\begin{figure*}[htbp]
	\centering
		\includegraphics[width=\textwidth]{../../FIGS/AgeSchedule.pdf}
	\caption{Age-schedule information for female and male halibut and assumed variability and relative abundance (transparency). Relative fecundity-at-age (fa, topleft), length-at-age (la), survivorship to age (lz, topright) assuming a fishing mortality rate of 0.2, landed value (pa), probability of dying due to fishing, and weight-at-age.}
	\label{fig:FIGS_AgeSchedule}
\end{figure*}

% subsection life_history_parameters_and_price_information (end)


% section methods (end)
%!TEX root = /Users/stevenmartell1/Documents/IPHC/SizeLimitz/docs/manuscript/SizeLimitz.tex
\section*{Results} % (fold)
\label{sec:results}

\subsection*{Growth} % (fold)
\label{sub:growth}

Size-at-age data in each of the regulatory areas has very marked differences in both the mean length-at-age, and the distribution of age-classes (Figure \ref{fig:FIGS_fig:lengthAgeFitbySex}). Male and female halibut in area 2A are fast growing by tend to have a much smaller asymptotic size than halibut sampled in other regulatory areas.  Moreover, the age-composition in area 2A is truncated relative to other regions, with very few fish older than 17 years.  The coefficient of variation in length-at-age is much higher in areas 2B and 2C, especially for females.  Estimated growth rates for female halibut in these two areas is nearly linear for younger ages, and on average older halibut in these regions are much larger in comparison to other regions with old female halibut.  Area 4B is also another anomaly in that the age-distribution is much older, especially for males, with many sampled individuals beyond age-20.  Additional details about the growth model and estimated model parameters are found in Appendix \ref{sec:estimation_of_growth}.

\begin{figure}[htbp]
	\centering
		\includegraphics[height=4in]{../../FIGS/fig:lengthAgeFitbySex.pdf}
	\caption{Length-at-age data by sex and regulatory area collected in the 2011 setline survey and corresponding estimated growth curves for each regulatory area.}
	\label{fig:FIGS_fig:lengthAgeFitbySex}
\end{figure}
% subsection growth (end)

\subsection*{Area-specific estimates of F$_{\rm{MSY}}$} % (fold)
\label{sub:area_specific_estimates_of_FMSY}

The relative equilibrium yield versus fishing mortality rate in each of the regulatory areas  (Figure \ref{fig:FIGS_fig:YeBase}) demonstrate the relative differences in expected yield based only on differences in halibut growth (size-at-age) in each of the regulatory areas.  For each of the regulatory areas shown in Figure \ref{fig:FIGS_fig:YeBase}, a minimum size limit of 81.3cm exists, steepness is fixed at an arbitrary value of 0.75, size-selectivity is the same in each area, and natural mortality rates are the same ($M=0.15$) for all areas.  The only biological difference between regulatory areas is the growth rate.  For each recruiting halibut in a specific regulatory area, the maximum yield per recruit would be obtained in area 2A.  Halibut in area 2A are very large at younger ages and more vulnerable to the fishing gear (selectivity) at a time when they are numerically more abundant.

In contrast, in Area 4C halibut obtain very large sizes, but the growth rate is much slower in comparison to 2A so fewer individuals are available to be harvested and less yield per recruit is obtained in this area.  The net result of this difference in growth rates is that estimate of F$_{\rm{MSY}}$ are lower in Area 4C relative to Area 2A.


\begin{figure}[htbp]
	\centering
		\includegraphics[height=4in]{../../FIGS/fig:YeBase.pdf}
	\caption{Relative equilibrium yield versus fishing mortality in the directed fishery assuming steepness $h=0.75$, a minimum size limit of 81.3cm, and no other sources of additional mortality. Vertical arrows pointing to the axis indicate the estimated F$_{\rm{MSY}}$ for each regulatory area.}
	\label{fig:FIGS_fig:YeBase}
\end{figure}

Equilibrium yield curves for each regulatory area under each of the 9 alternative scenarios are shown in Figure \ref{fig:FIGS_fig:YeALL} and the corresponding estimates of optimal exploitation rates ($1-exp(-F_{\rm{MSY}})$) are summarized in Table \ref{table:Umsy}.  Relative to the current harvest rate policy of 21.5\% and 16.125\%, estimates of optimal exploitation rates for area 2B and 2C are below the current 21.5\% value.   However, recall that this is based on the assumption of a Beverton-Holt stock recruitment relationship with a steepness value set at an arbitrary level of 0.75.  This arbitrary value was arrived at because it was consistent with the previous harvest rate policy, which was developed using two average density-independent recruitment values based on previous closed area assessment models.  The utility of S1 is to serve as a baseline in which to compare impacts of alternative size limits, and model assumptions,  on the estimates of optimal exploitation rates that would maximize the average long-term yield in each of the statistical areas.


\begin{figure}[htbp]
	\centering
		\includegraphics[width=\textwidth]{../../FIGS/fig:YeALL.pdf}
	\caption{Relative equilibrium yield versus fishing mortality for all 9 scenarios described in Table \ref{table:Scenarios}, where the different line colors denotes Regulatory area.}
	\label{fig:FIGS_fig:YeALL}
\end{figure}

\begin{table}
	\caption{Estimates of optimal exploitation rates for each regulatory area and scenario combination. Scenario descriptions are found on page \pageref{sub:scenarios}.}
	\label{table:Umsy}
	\begin{center}
		\begin{tabular}{c|ccccccccc}
		\hline
		Scenario & 2A & 2B & 2C & 3A & 3B & 4A & 4B & 4C & 4D\\
		\hline
		S1& 0.248 &0.197 &0.180 &0.297 &0.318 &0.260 &0.180 &0.176 &0.300\\
		S2& 0.248 &0.221 &0.221 &0.300 &0.321 &0.271 &0.225 &0.237 &0.304\\
		S3& 0.221 &0.180 &0.163 &0.256 &0.275 &0.229 &0.168 &0.163 &0.260\\
		S4& 0.201 &0.163 &0.151 &0.229 &0.245 &0.209 &0.155 &0.151 &0.233\\
		S5& 0.168 &0.137 &0.133 &0.172 &0.185 &0.168 &0.137 &0.133 &0.176\\
		S6& 0.180 &0.133 &0.120 &0.193 &0.205 &0.163 &0.124 &0.111 &0.193\\
		S7& 0.213 &0.176 &0.168 &0.252 &0.275 &0.233 &0.172 &0.172 &0.256\\
		S8& 0.260 &0.209 &0.197 &0.289 &0.311 &0.264 &0.197 &0.189 &0.293\\
		S9& 0.155 &0.129 &0.120 &0.180 &0.193 &0.163 &0.124 &0.120 &0.180\\
		\hline
		\end{tabular}
	\end{center}
\end{table}

%Imposing a maximum size limit.
The maximum size limit scenario results in slight increases in the estimates of $F_{\rm{MSY}}$ in areas where halibut grow to a sufficiently large size to benefit from such protection (Table \ref{table:Umsy}).  Imposing a maximum size limit does not result in any yield benefits in any of the regulatory areas (Scenario S2, Table \ref{table:MSY}).  In fact, in Areas 2A and 3B, there is a very small probability that an individual halibut would survive and grow to surpass the upper legal size limit of 140cm. There is only a minor improvement in the relative spawning biomass in areas 2A and 3B associated with a maximum size limit of 140cm (Table \ref{table:Bmsy}).  Whereas, there is a further reduction in the spawning biomass depletion in areas where halibut grow beyond the 140cm maximum size limit and fishing at $F_{\rm{MSY}}$, and the amount of spawning biomass reduction is proportional to the discard mortality rates.


%Removing size limits.
If size-limits were removed altogether, and there is no change in the size-selectivity of the commercial fishery, then estimates of $F_{\rm{MSY}}$ would have to be reduced (S3, Table \ref{table:Umsy}) in order to compensate for the increased total mortality rate associated with retaining fish smaller than 81.3cm.  Relative increases in overall yield do occur with the removal of the minimum size limits, as the yield per recruit in each area increases, with the exception of Regulatory area 4B.  However, this modest increase in overall yield does come at the expense of reducing spawning stock biomass, as well as, the average size of landed fish would also be reduced.

Scenario 4 represents a shift in the commercial selectivity towards smaller fish, and the net impact of this shift is a reduction in the $F_{\rm{MSY}}$ values for each regulatory area, as well as, decreases in the overall landed yield (Tables \ref{table:Umsy} and \ref{table:MSY}).  Recall that this scenario was run with the current minimum size limit of 81.3cm in place and serves to show that minimum size-limits alone does not afford protection of spawning stock biomass if discard mortality rates are greater than 0.  Although, corresponding increases in spawning biomass are observed in Table \ref{table:Bmsy} for scenario 4, this increase owes to the reduction in $F_{\rm{MSY}}$ that would be required to maximize yield if selectivity were to shift towards smaller fish.

\begin{table}
	\caption{Relative change in yield by regulatory area in comparison to S1 (status quo) for each of th alternative scenarios while fishing at rates defined in Table \ref{table:Umsy} (MSY based fishing mortality).}
	\label{table:MSY}
	\begin{center}
		\begin{tabular}{c|ccccccccc}
		\hline
		Scenario & 2A & 2B & 2C & 3A & 3B & 4A & 4B & 4C & 4D\\
		\hline
		S1&  0.00&  0.00&  0.00&  0.00&  0.00&  0.00&  0.00&  0.00&  0.00 \\
		S2&  0.00& -0.17& -0.27& -0.01&  0.00& -0.04& -0.33& -0.41& -0.01 \\
		S3&  0.65&  0.30&  0.11&  0.85&  0.78&  0.28& -0.01&  0.04&  0.45 \\
		S4& -0.04& -0.03& -0.02& -0.14& -0.15& -0.09& -0.04& -0.04& -0.14 \\
		S5& -1.41& -0.84& -0.69& -1.41& -1.44& -1.12& -0.71& -0.61& -1.47 \\
		S6& -0.72& -0.83& -1.00& -0.86& -0.89& -0.98& -1.00& -1.00& -0.91 \\
		S7&  0.32&  0.38&  0.48&  0.24&  0.24&  0.40&  0.45&  0.44&  0.26 \\
		S8&  1.10&  0.72&  0.66&  0.57&  0.55&  0.63&  0.69&  0.49&  0.60 \\
		S9& -1.13& -0.76& -0.69& -0.87& -0.88& -0.82& -0.78& -0.60& -0.92 \\
		%
	   %S1& 7.55 &5.41 &4.99 &5.72 &5.90 &5.81 &5.83 &4.61 &6.35\\
	   %S2& 7.54 &5.25 &4.72 &5.71 &5.89 &5.76 &5.49 &4.20 &6.34\\
	   %S3& 8.20 &5.72 &5.10 &6.58 &6.68 &6.09 &5.82 &4.65 &6.80\\
	   %S4& 7.50 &5.39 &4.97 &5.58 &5.74 &5.72 &5.79 &4.56 &6.21\\
	   %S5& 6.14 &4.57 &4.29 &4.31 &4.46 &4.68 &5.12 &3.99 &4.88\\
	   %S6& 6.83 &4.59 &3.99 &4.86 &5.01 &4.82 &4.83 &3.61 &5.44\\
	   %S7& 7.87 &5.79 &5.47 &5.96 &6.13 &6.20 &6.28 &5.05 &6.61\\
	   %S8& 8.65 &6.13 &5.64 &6.29 &6.44 &6.43 &6.52 &5.10 &6.95\\
	   %S9& 6.42 &4.66 &4.30 &4.85 &5.01 &4.98 &5.05 &4.00 &5.43\\
		\hline
		\end{tabular}
	\end{center}
\end{table}

The effects of other discard mortality, from non-directed fisheries, also plays a role in harvest policy calculations.  In scenario 5, a constant total harvest (i.e., bycatch from a trawl fishery) was imposed as an increasing fishing mortality rate with increasing directed $F_e$.  In this scenario, a constant level of bycatch results in a dramatic reduction in overall yield in the directed fishery (Table \ref{table:MSY}) and a reduction in the optimal harvest rate ($F_{\rm{MSY}}$) that would produce the maximum sustainable yield in the directed fishery.  This is similar to discarding undersized fish, but in this case the mortality rate imposed by the bycatch fishery increases with declining stock size.  Whereas, in the directed fishery, annual catches would scale down with reductions in exploitable biomass, and fishing mortality rates would scale down if the spawning biomass falls below B$_{30\%}$.  It should also be noted that the opposite result would occur where $F_{\rm{MSY}}$ estimates would increase if efforts were made to reduce bycatch in non-directed fisheries.  

%Bmsy table
\begin{table}
	\caption{Relative spawning biomass while fishing at $F_{\rm{MSY}}$ for each scenario and regulatory area.}
	\label{table:Bmsy}
	\begin{center}
		\begin{tabular}{c|ccccccccc}
		\hline
		Scenario & 2A & 2B & 2C & 3A & 3B & 4A & 4B & 4C & 4D\\
		\hline		%
		S1&$23.8$&$25.5$&$25.8$&$26.0$&$26.8$&$26.9$&$27.6$&$28.3$&$27.1$\tabularnewline
		S2&$23.9$&$24.4$&$24.2$&$25.7$&$26.6$&$26.2$&$25.5$&$25.4$&$26.9$\tabularnewline
		S3&$22.6$&$24.5$&$25.4$&$23.0$&$23.5$&$25.0$&$27.0$&$27.6$&$24.4$\tabularnewline
		S4&$24.8$&$26.1$&$26.7$&$27.5$&$28.3$&$27.7$&$27.8$&$28.8$&$28.5$\tabularnewline
		S5&$26.0$&$27.7$&$26.8$&$29.4$&$30.1$&$29.0$&$28.4$&$29.2$&$30.4$\tabularnewline
		S6&$24.1$&$25.9$&$25.7$&$26.0$&$26.6$&$27.2$&$26.7$&$28.2$&$27.3$\tabularnewline
		S7&$25.7$&$27.7$&$28.0$&$28.7$&$29.3$&$29.0$&$28.7$&$30.0$&$29.7$\tabularnewline
		S8&$20.5$&$22.4$&$22.4$&$24.3$&$25.1$&$24.7$&$24.1$&$25.6$&$25.6$\tabularnewline
		S9&$29.2$&$29.5$&$29.7$&$30.4$&$31.3$&$30.8$&$30.7$&$31.8$&$31.9$\tabularnewline
		\hline
		\end{tabular}
	\end{center}
\end{table}
% subsection area_specific_estimates_of_f__{msy_ (end)

\subsection*{Sensitivity to assumed parameter values} % (fold)
\label{sub:sensitivity_to_assumed_parameter_values}
Scenarios 6 and 7 examine how sensitive MSY-based reference points are to estimates of natural mortality rates.  The current assessment model assumes natural mortality is independent of size/age.  These two scenarios examine a size-effect in natural mortality.  In general, if $M$ is size/age independent, the increasing $M$ results in increases in the estimates of $F_{\rm{MSY}}$, and vice versa \citep{WalMart2004}.  Also increases in $M$ results in a decrease in MSY and the spawning biomass at $F_{\rm{MSY}}$.  If $M$ is size-dependent and decreases with size, estimates of $F_{\rm{MSY}}$  also decrease, and vice versa.  Natural mortality also plays a role in the general scaling of MSY; as fewer older fish are available for harvest due to high natural mortality rates, then the value of MSY decreases.  Hence mis-specification of $M$ can lead to biased estimates of other reference points (e.g., SB$_{100}$).

Lastly, estiamtes of $F_{\rm{MSY}}$ are very sensitive to the steepness of the stock-recruitment relationship.  With increasing steepness the corresponding estimates of $F_{\rm{MSY}}$ also increase, and vice versa.  Moreover, the resilience of the stock to over-fishing decreases with decreasing values of steepness, and the overall yield declines as well as you get fewer recruits per unit of spawning biomass.

% subsection sensitivity_to_assumed_parameter_values (end)

\subsection*{Wastage in the directed fishery} % (fold)
\label{sub:wastage_in_the_directed_fishery}
\begin{figure}[htbp]
	\centering
		\includegraphics[width=\textwidth]{../../FIGS/fig:PercentWastage.pdf}
	\caption{Percent wastage in the directed fishery versus long-term equilibrium fishing mortality rate for each regulatory area. Scenario 1 consists of a minimum size limit of 81.3cm, scenario 2 includes an additional maximum size limit of 140cm, and scenario 4 consists of a 81.3cm minimum size limit and commercial selectivity shifting 10cm towards smaller fish.}
	\label{fig:FIGS_fig:PercentWastage}
\end{figure}

% subsection wastage_in_the_directed_fishery (end)

\subsection*{Changes in mean weight-at-age} % (fold)
\label{sub:changes_in_mean_weight_at_age}


% subsection changes_in_mean_weight_at_age (end)




% section results (end)


%!TEX root = /Users/stevenmartell1/Documents/IPHC/SizeLimitz/docs/manuscript/SizeLimitz.tex


\section*{Discussion} % (fold)
\label{sec:discussion}

%% Summary
Factors that affect estimates of optimal harvest rates come in two general forms: (1) biological components that define the underlying productivity of the stock, and (2) fishery components that affect size-at-entry and size-specific mortality. The former cannot be directly managed but must be taken into consideration in harevst policy, especially if there are temporal changes in stock productivity.  The latter can be directly controlled through a variety of tools that limit gear specifications, size limits, or even areas fished, to control size-at-entry into the fishery and reduce post release mortality rates.  In the case of Pacific halibut, there have been recent changes in size-at-age and there is also considerable variation in the size-at-age among the regulatory areas. Optimal harvest rate calculations for each of the regulatory areas will have to be updated frequently owing to continued changes, and differences, in size-at-age among regulatory areas.  Also, any changes in fisheries operations or changes in fishery regulations (e.g., change in size limits or bycatch) will also affect optimal harvest calculations.  In general, the underlying parameters that define the harvest policy should be updated on a routine, or even annual, basis to ensure that biological and technological changes are taken into account in updating harvest policy.

In this paper, we examined how changes in stock productivity and size-selectivity interact with estimates of fishing mortality rates that maximize long-term sustainable yields.  This was approximated using an age- and sex-structured equilibrium model conditioned on regulatory area size-at-age data from the 2011 setline survey.  Previous harvest policy development was based on a stochastic simulation model for Pacific halibut that is now dated because of: (a) continued changes in size-at-age for Pacific halibut, (b) a transition from area-based assessments to a coast-wide model, (c) recent recognition that fishery and survey selectivity must vary over time in the coast-wide assessment due to changes in the distribution of the stock, and (d) reduction in the bycatch levels in non-directed fisheries.  There have been previous equilibrium-based models for the development of harvest policy for Pacific halibut \citep[e.g.,][]{clark1995re}, but these models did not explicitly consider the effects of wastage and bycatch on optimal harvest rates.  Moreover, previous analyses were also limited to the core halibut areas (2B, 2C and 3A).  


At this point in time, this analysis should be considered a work in progress for the sole reason that a key population parameter (i.e., steepness) that defines the underlying stock productivity is not yet available for the new coast-wide assessment.  The steepness parameter was arbitrarily set at a value of 0.75 and was chosen because it resulted in estimates of $F_{\rm{MSY}}$ that are similar to those obtained by \cite{clark2006assessment}.  Nevertheless, the relative changes in $F_{\rm{MSY}}$ among regulatory areas based on differences in growth rates would not differ if a reliable estimate of steepness was available.  Also critical to optimal harvest rate calculations is the area-based selectivity of fishing gears that harvest Pacific halibut.  At present, the coast-wide assessment model assumes size-based selectivity for each fishing gear does not vary by regulatory area.  Next to differences in size-at-age, regulatory area differences in selectivity-at-age relative to maturity-at-age will also have a large impact on estimates of optimal exploitation rates.  The previous closed-area assessment models estimated marked differences in selectivity among regulatory areas 2B and 3A \citep{clark2006assessment}. The transition to a coast-wide model introduced a new assumption that size-based selectivity is the same for all regulatory areas.

 One issue, that has not been examined here, and has very important harvest policy implications is the initial recruitment and movement of halibut among regulatory areas. Area-specific optimal harvest rates are sensitive to movement of halibut among regulatory areas.  There has been a considerable effort in this regard to understand the movement of halibut \citep[e.g.,][]{loher2006seasonal,webster2009analysis} and what the potential implications are for harvest policy \citep{valero2009exploring,valero2010effect}.  In general, areas with a net migration loss are nearly equivalent to having a higher natural mortality rate in a closed area model.  The harvest policy implications in such a case would be to harvest at a higher rate, but the total removals would would scale down.  The opposite is true for an area that has a net migration increase, harvest at a lower rate, but the scale of the harvest increases due to immigration to the area.  If however, the objective is to maximize the yield from all areas combined, then the optimal harvest rate calculations are much more complex involving dispersal kernels for new recruits and migration transition matrices.  Under such circumstances it is not possible to make generalized statements about how optimal harvest rates would change because the answer depends on the relative migration coefficients among the regulatory areas.  \cite{valero2010effect} had made progress in this area and their early conclusions suggested that harvest policies to the north of area 2 would have fairly severe implications for area 2 itself due to downstream migration of halibut into this area.


It is intuitive to think that imposing a maximum size limit would afford protection to sexually mature fish that grow beyond the size limit and that this would lead to an increase in spawning biomass (or reduce the level of depletion).  This does occur, but only if there is a very low discard mortality rate associated with releasing fish.  In the case examined here, with 140cm size limit and fishing at $F_{\rm{MSY}}$, the spawning biomass in each regulatory area remains nearly the same or  declines in comparison to no maximum size limit.  The reason for this decline is related to a discard mortality rate of 0.16 per year, and under MSY-based harvest policies, values of $F_{\rm{MSY}}$ would increase in areas where halibut grow to sufficient size and a maximum size limit is used in the harvest policy.

Shifts in the directed commercial selectivity schedule towards smaller halibut pose a conservation concern if the discard mortality rate is greater than 0, even if minimum size limits are in place. If individual IFQ holders are not accountable for their discard mortality of undersized fish, then fishing can continue until their quota is filled with legal-sized halibut.  If there is a shift towards catching smaller fish, or the probability of capturing a legal-size fish in a given area is low, then the corresponding increase in discard mortality  results in a higher overall total mortality rate that may not be accounted for if the shift in selectivity goes undetected.  This is of particular concern in the current assessment of Pacific halibut, where the wastage calculation assumes the commercial fishery selectivity is the same as the top 33\% of the setline survey WPUE \citep{gilroy2009wastage}.  Moreover, estimated commercial selectivity is based on composition data from port samples, not on fish sampled on the boats at sea when the gear is being retrieved. In other words, the wastage calculation in the directed fishery is based on a tenuous assumption about how the commercial gear selects fish less than 81.3cm.

The harvest policy implications of undetected changes in selectivity and estimates of optimal exploitation rates are somewhat insensitive if the discard mortality rates are low or even negligible.  If under-sized fish are handled with extreme care such that survival rates are near 100\%, then the previous discussion about uncertainty in commercial selectivity for under-size fish is moot.  Moreover, the estimate of wastage would consist only of lost or abandoned gear.  However, if the release mortality rates are appreciable, then estimates of optimum harvest rates must also include release mortality associated with size-limits  \citep{goodyear1993spawning,coggins2007ecm}.  In general,  as the size limit increases the optimum fishing rate that would maximize yield increases exponentially.  This relationship is also the same for the fishing mortality rate that would deplete the spawning biomass to some target level.  The exponential increase in $F_{\rm{MSY}}$ occurs when individuals have had at least one chance to spawn before they become vulnerable to fishing.  \cite{pineiii2008car} demonstrated that as the release mortality rates increase, the potential of a minimum size limit to hedge against overfishing decreases as the release mortality rates increase.

If the observed changes in size-at-age are a result of cumulative size-selective fishing, then the largest changes in size-at-age would be expected in areas with higher fishing mortality rates.  In closed populations, the variance in size-at-age for older fish is expected to decrease with increasing fishing mortality rates.  Areas 2B and 2C are thought to have fairly high fishing mortality rates based on the results of the stock assessments and biomass apportionment \citep{Hare2012Rara}.  Based on the size-at-age data from the setline survey, the largest variance in size-at-age is found in Areas 2B and 2C, suggesting that these areas are heavily influence by migration into the area.


% concluding paragraph?
In the very near future, the Pacific halibut assessment model is likely to evolve to a more implicit spatial representation where estimated selectivities by regulatory area may differ (Ian Stewart, pers comm).  Given that steepness is unknown, and selectivity likely differs by regulatory area, it is not recommended to change the current harvest policy until, at a minimum, these two issues have been addressed.  Preferably, the full suite of biological factors, including dispersal and migration, and factors that affect selectivity would be included in the harvest policy analysis.

% section discussion (end)


\addcontentsline{toc}{section}{References}
\bibliographystyle{apalike}
\bibliography{$HOME/Documents/ARTICLES/Articles-1}

\appendix
\input{./Appendix/Growth}
%\input{./Appendix/F_ReferencePoints}

\end{document}
