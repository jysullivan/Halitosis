%
%  SizeLimitz
%
%  Created by Martell on 2012-09-05.
%  Copyright (c) 2012 UBC Fisheries Centre. All rights reserved.
%
\documentclass[12pt]{article}

% Use utf-8 encoding for foreign characters
\usepackage[utf8]{inputenc}

% Setup for fullpage use
\usepackage{fullpage}

% Bibliography
\usepackage[round]{natbib}	

% Uncomment some of the following if you use the features
%
% Running Headers and footers
%\usepackage{fancyhdr}

% Multipart figures
%\usepackage{subfigure}

% More symbols
\usepackage{amsmath}
\usepackage{amssymb}
\usepackage{latexsym}

% Surround parts of graphics with box
\usepackage{boxedminipage}

% Package for including code in the document
\usepackage{listings}

% If you want to generate a toc for each chapter (use with book)
\usepackage{minitoc}

% This is now the recommended way for checking for PDFLaTeX:
\usepackage{ifpdf}

%\newif\ifpdf
%\ifx\pdfoutput\undefined
%\pdffalse % we are not running PDFLaTeX
%\else
%\pdfoutput=1 % we are running PDFLaTeX
%\pdftrue
%\fi

\ifpdf
\usepackage[pdftex]{graphicx}
\else
\usepackage{graphicx}
\fi
\title{Size Limits, Post Release Mortality, and Cumulative Effects of Size Selective Fishing on Pacific Halibut}
\author{ Steven Martell\\ IPHC }

\date{2012-09-05}

\begin{document}

\ifpdf
\DeclareGraphicsExtensions{.pdf, .jpg, .tif}
\else
\DeclareGraphicsExtensions{.eps, .jpg}
\fi

\maketitle


\begin{abstract}
	Harvesting fish is inherently a size-selective process where faster growing individuals recruit to the fishery at a younger age than slower growing individuals.  As a consequence of continuous fishing the size composition of each cohort may be eroded where only slower growing, smaller individuals, remain in the population.  Limiting the size range of fish that can be harvested can protect spawning biomass and allow for population renewal, if and only if, a fraction of the captured individuals that are outside the legal size range survive the harvesting process.  Release mortality (or discard mortality) is the fraction of individuals that die after capture and release.  If the release mortality rates are substantially greater than 0, limiting the legal size range could have a negative impact on potential spawning biomass if fishing mortality rates are sufficiently high.  Moreover abundant cohorts will experience above average total mortality rates at younger ages/sizes with minimum size limits in comparison to no size limits.  Cumulative effects of size-selective fishing give the appearance of declining size-at-age.  This effect is amplified with higher release mortality rates and minimum size limits that are much greater than the size-at-entry to the fishery.
	
\end{abstract}

%!TEX root = /Users/stevenmartell1/Documents/IPHC/SizeLimitz/docs/manuscript/SizeLimitz.tex
\section{Notes} % (fold)
\label{sec:notes}

\begin{itemize}
	\item Run growth model and calculate F0.1  using only parametric uncertainty in growth.
	\item Calculate F0.1 using parametric uncertainty in growth and a prior distribution for natural mortality.
\end{itemize}


% section notes (end)

\section*{Introduction} % (fold)
\label{sec:introduction}

The current harvest strategy for Pacific halibut includes the use of minimum size limits along with a constant exploitation rate policy to determine annual total catches for commercial and non-commercial fisheries.  Previous studies have examined the impacts of alternative size limits on the yield per recruit and spawning biomass per recruit using steady-state or equilibrium models \citep{clark1995re}, and more recently using a simulation modelling approach \citep{clark2006assessment}.  The current harvest policy for Pacific halibut utilizes a minimum legal size limit of 81.3cm (or 32 inches) total length.  Previous studies have not considered the impacts of size-selective fishing and release mortality rates on optimal harvest policies for Pacific halibut.  Using a minimum size limit, faster growing individuals from a given cohort would experience higher total mortality rates over their lifetime as they recruit to the legal size at a younger age in comparison to slower growing individuals. The cumulative effect over time imposes a higher total mortality rate on faster growing halibut and surviving individuals in the population are biased towards slower growing individuals \citep{Taylor2005}. The cumulative effects of size-selective fishing can give the appearance of declining mean weight-at-age in the population.

Previous yield per recruit and size limit interactions for Pacific halibut did not consider the effects of discard mortality rates. The underlying assumption in this case is that all fish captured and released will survive. \cite{coggins2007ecm} and \cite{pineiii2008car} demonstrated that for a given size limit, increases discard mortality can lower the realized yield per recruit. Moreover, non-zero discard mortality rates would reduce the potential benefit, if any, of an upper size limit as it would afford less opportunity to survive to the upper size limit, and reduce survivorship of captured individuals that not of legal size.

Maximizing the yield per recruit, or maximizing the total equilibrium yield is not always desirable from an economic perspective.  In the case of Pacific halibut, there is a differential price structure for the size of fish landed, where larger fish command much higher per pound prices than smaller fish. In such case, it may be more desirable to fish at rates lower than $\rm{F_{msy}}$, or F$_{0.1}$ in order to maximize the economic value.

In this paper, I develop an equilibrium model that address both the cumulative effects of size-selective fishing and discard mortality rates for Pacific halibut.  Yield per recruit, spawning biomass per recruit, equilibrium yield, depletion and landed value per recruit are compared using five alternative harvest policy scenarios: (1) the current size size limit of 81.3 cm (or 32 inches), (2) no size limits, (3) a 60 cm (or 26 inch) size limit, and (4) a slot limit with lower and upper bounds of 81.3--150 cm, and (5) a slot limit with lower and upper bounds of 60--150 cm.  Each of these five harvest policy scenarios are examined under alternative hypotheses about changes in size-at-age and changes in size-selectivity in the commercial fishery.




% section introduction (end)

%!TEX root = /Users/stevenmartell1/Documents/IPHC/SizeLimitz/docs/manuscript/SizeLimitz.tex
\section*{Methods} % (fold)
\label{sec:methods}

\subsection*{Equilibrium model} % (fold)
\label{sub:equilibrium_model}

For ease of describing the age-structure equilibrium model I ignore the cumulative effects of size-selective fishing and size limits. In an equilibrium setting the annual yield is actually a sum over all ages of the fraction of individuals that die due to fishing multiplied by the total number or biomass of individuals available for harvest.  Thus the equilibrium yield equation can be written as:
\begin{equation}\label{eq:Y_e}
	Y_e = \sum_{a=1}^\infty \frac{B_a F_a [1-\exp(-M_a-F_a)]}{M+F_a}
\end{equation}
where $F_a$ is the age-specific fishing mortality rate which can be parsed as $F_e v_a$, where $v_a$ is the age-specific fraction that is vulnerable to fishing mortality (also termed selectivity in models that do not distinguish between landed and discarded fish).  Biomass at age ($B_a$) is defined as the numbers-at-age ($N_a$) times the average weight-at-age ($w_a$).  Assuming steady-state conditions this can be expressed as the product of recruitment, survivorship and average weight-at-age.  Assuming unfished conditions (i.e., $F_e=0$), survivorship to a given age is given as:
\begin{equation}\label{eq:unfished_survivorship}
	l_a =\begin{cases} 1, & a=1 \\ l_{a-1} \exp(-M_a), &a>1\end{cases}, 
\end{equation}
and survivorship under fished conditions is given by:
\begin{equation}\label{eq:fished_survivorship}
	\acute{l}_a =\begin{cases} 1, & a=1 \\ \acute{l}_{a-1} \exp(-M_a-F_e v_a), &a>1\end{cases}, 
\end{equation}
The age-specific biomass is given as:
\begin{equation} \label{eq:B_a}
	B_a = R_o l_a w_a.
\end{equation}

Substituting \eqref{eq:B_a} into  \eqref{eq:Y_e} and parsing the fishing mortality rate components yields the following expression
\begin{equation}\label{eq:Y_e2}
	Y_e = F_e R_e \sum_{a=1}^\infty \frac{l_a,w_a v_a [1-\exp(-M_a-F_e v_a)]}{M+F_e v_a},
\end{equation}
where $R_e$ is the equilibrium recruitment obtained under a fishing mortality rate $F_e$. The summation term in \eqref{eq:Y_e2} represents the yield per recruit ($\phi_q$), and the yield equation simplifies to:
\begin{equation}\label{eq:Y_e3}
	Y_e = F_e R_e \phi_q.
\end{equation}

For a given equilibrium fishing mortality rate $F_e$, the equilibrium recruitment is a function of the available spawning biomass relative to the unfished spawning biomass. For the Beverton-Holt model, this can be expressed as
\begin{equation}\label{eq:R_e}
	R_e = \frac{R_o (\kappa-\phi_e/\phi_f)}{\kappa -1} 
\end{equation}
where the spawning biomass per recruit $\phi_e$ and $\phi_f$ for unfished and fished conditions, respectively, is based on the survivorship and mature female weight-at-age,  or fecundity-at-age ($f_a$).  Two leading parameters are the unfished age-1 recruits $R_o$, which serves the purpose of providing the scale of the population, and the recruitment compensation parameter $\kappa$ which is defined as the relative improvement in juvenile survival rate as the spawning biomass tends to zero (for the Beverton-Holt model this can be derived from steepness as $\kappa= 4h/[1-h]$).  Spawning biomass per recruit is given by:
\begin{eqnarray}
	\phi_e = \sum_{a=1}^\infty l_a f_a\label{eq:phi_e}\\
	\phi_f = \sum_{a=1}^\infty \acute{l}_a f_a\label{eq:phi_f}
\end{eqnarray}
Note that it is not necessary to have absolute estimates of fecundity and the units cancel out in the $\phi_e/\phi_f$ ration in \eqref{eq:R_e}. What is important is the relative egg contribution by age, and here it is assumed that fecundity is proportional to mature female body weight.

Based on equations \ref{eq:Y_e}--\ref{eq:phi_f} it is now possible to calculate the equilibrium yield given estimates of the following parameters: $\Theta = \{R_o, \kappa, M_a, f_a, w_a, v_a\}$.
% subsection equilibrium_model (end)

\subsection*{Including release mortality} % (fold)
\label{sub:including_release_mortality}
The equilibrium model described in the previous section only considers the case in which all fish captured for a unit of fishing mortality $F_e$ are removed from the population and does account for cases in which some fish captured will be discarded because they are not within the legal size range.  To include the effects of post release mortality associated with size limits, the vulnerability age-schedule ($v_a$) has to be modelled as as a joint probability where there the probability of dying due to fishing is based on the probability of capture and being retained times the probability of being captured, released, and dying after release.  This simple joint probability is as follows:
\begin{equation} \label{eq:v_a}
	v_a = v_c[v_r + (1-v_r)\lambda]
\end{equation}
where $v_a$ is the age-specific mortality rate associated with a unit of fishing mortality, $v_c$ is the age-specific probability of being captured by fishing gear, $v_r$ and $(1-v_r)$ is the age specific retention (release) probability, and $\lambda$ is the probability of dying after being discarded.

To implement the effects of size limits and post-release mortality rates on the equilibrium yield calculations defined in the previous section, we simply substitute \eqref{eq:v_a} for all the $v_a$ terms in equations \ref{eq:fished_survivorship} and \ref{eq:Y_e3} above.
% subsection including_release_mortality (end)

\subsection*{Cumulative effects of size-selective fishing} % (fold)
\label{sub:cumulative_effects_of_size_selective_fishing}
To account for variation in growth and represent the cumulative effects of size-selective fishing, the population is divided into a number of distinct groups ($G$) that each have a unique asymptotic length ($l_{\infty,g}$).  In an unfished population, the assumption is that the distribution of asymptotic lengths is normally distributed with a coefficient of variation of 0.1.  The proportion of recruitment to each of these $G$ groups is assumed to be normally distributed with 99.7\% of all individuals falling within 3 standard deviations of the mean asymptotic length.  There are no assumptions about the composition of the spawning stock biomass and recruitment into each of these groups (i.e., no genetic selection effects due to fishing is assumed), and irrespective of spawning stock size, recruitment to each of these groups follows the same normal distribution.

The per recruit functions described in the previous equations are then modified to include both the age and size effect.  For example the spawning biomass per recruit described in \eqref{eq:phi_f} is now calculated as:
\begin{equation}
	\phi_{f} =\sum_{g=1}^G p_g \sum_{a=1}^\infty \acute{l}_a f_a\label{eq:phi_fg}
\end{equation}
where $p_g \sim N(0,1)$ and computed over 11 discrete intervals from -3 to 3.  In other words, the equilibrium population consists of 11 discrete sub-populations that differ only in their asymptotic lengths and the relative abundance of each sub-population follows a normal distribution.
% subsection cumulative_effects_of_size_selective_fishing (end)

\subsection*{Life-history parameters and price information} % (fold)
\label{sub:life_history_parameters_and_price_information}
For the following analysis, the assumed life-history parameters are listed in Table 1

\begin{table}
	\caption{Life-history parameters used in the equilibrium model.}
	\label{table:Life_history_pars}
	\begin{center}
	\begin{tabular}{lcccc}
		\hline
		 &  & & \multicolumn{2}{c}{Sex-specific}\\
		Parameter         & Symbol     & Value & Female & Male \\
		\hline
		Natural mortality & $M$        &       &   0.15 & 0.18 \\
		Asymptotic length & $l_\infty$ &       &   145  & 110  \\
		Metabolic rate    & $k$        &       &   0.10 & 0.12 \\
		\hline
		
		\multicolumn{3}{l}{Selectivity coefficients}
		 Size (cm) & Female & Male\\
		\hline
		&&  50 & 0.0000& 0.0000\\
		&&  60 & 0.0000& 0.0040\\
		&&  70 & 0.0603& 0.0567\\
		&&  80 & 0.3009& 0.2814\\
		&&  90 & 0.6303& 0.5855\\
		&& 100 & 0.9139& 0.8356\\
		&& 110 & 1.0000& 1.0000\\
		&& 120 & 1.0000& 1.0000\\
		\hline
	\end{tabular}
	\end{center}
\end{table}


\begin{figure*}[htbp]
	\centering
		\includegraphics[width=\textwidth]{../../FIGS/AgeSchedule.pdf}
	\caption{Age-schedule information for female and male halibut and assumed variability and relative abundance (transparency). Relative fecundity-at-age (fa, topleft), length-at-age (la), survivorship to age (lz, topright) assuming a fishing mortality rate of 0.2, landed value (pa), probability of dying due to fishing, and weight-at-age.}
	\label{fig:FIGS_AgeSchedule}
\end{figure*}

% subsection life_history_parameters_and_price_information (end)


% section methods (end)

\addcontentsline{toc}{section}{References}
\bibliographystyle{apalike}
\bibliography{$HOME/Documents/ARTICLES/Articles-1}


\end{document}
