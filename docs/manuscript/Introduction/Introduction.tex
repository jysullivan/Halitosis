%!TEX root = /Users/stevenmartell1/Documents/IPHC/SizeLimitz/docs/manuscript/SizeLimitz.tex
% \section{Notes} % (fold)
% \label{sec:notes}
% 
% \begin{itemize}
% 	\item Run growth model and calculate F0.1  using only parametric uncertainty in growth.
% 	\item Calculate F0.1 using parametric uncertainty in growth and a prior distribution for natural mortality.
% \end{itemize}







% section notes (end)

\section*{Introduction} % (fold)
\label{sec:introduction}
The current harvest objectives for Pacific halibut are to achieve a high level of yield while at all times maintaining healthy female spawning stock biomass to ensure long-term sustainability. The current harvest policy intends to achieve a desired exploitation rate of 21.5\% in regulatory Areas 2A, 2B, 2C, and 3A, and a lower exploitation rate of 16.125\% in Areas 3B, 4A, 4B, and 4CDE.  These harvest rates were determined using a simulation modeling approach based on the results of closed area assessments in Areas 2B, 2C and 3A \citep{clark2006assessment} and subsequent considerations of surplus production and yield per recruit calculations in western regulatory areas, as well as, a revised definition of exploitable biomass.  Since that time, there have been several significant changes to the assessment model and regulatory area management that warrant a review and update to the current harvest policy for Pacific halibut.  Most significant of these aforementioned changes was the transition from several area-based assessments to a single coast-wide assessment model and  apportionment. There have also been changes in mean size-at-age and fisheries selectivity for Pacific halibut, both of which have implications for optimal harvest rate calculations.

The definition of optimal harvest rate can be broadly defined as the harvest rate that would lead to maximizing the long-term yield.  Optimal harvest rates can be defined in terms of specific metrics; for example, the harvest rate that maximizes the yield per recruit (F$_{\rm{YPR}}$), or the harvest rate that reduces the spawning biomass per recruit to some specified level (F$_{\rm{SPR}}$), or the harvest rate that maximizes the long-term sustainable yield (F$_{\rm{MSY}}$).  Each of these alternative harvest rate metrics requires similar biological and fisheries selectivity information, with the exception of the F$_{\rm{MSY}}$ which requires additional information about the relationship between spawning stock biomass and recruitment. For the purpose of this report, the harvest rate that maximizes long-term sustainable yield is what is implied with the term optimal harvest rate and is denoted herein by F$_{\rm{MSY}}$.

Factors that affect the optimal harvest rate calculation fall into two general categories: 1) biological properties that define the underlying productivity of the stock, and 2) fishing related properties associated with target and non-target fisheries that catch halibut.  Important biological parameters that define the optimal harvest rate include: growth parameters and the variance in size-at-age, maturity- and fecundity-at-age, natural mortality rates, the steepness of the stock-recruitment relationship,  and migration rates among the various fishing grounds \citep{Beddington2005}. Although  dispersal and movement is important for determining optimal harvest rates, it is often ignored under the assumption of a unit stock (area is sufficiently large enough to ignore dispersal and migration). The vulnerability of fish to fishing gear, also known as selectivity, is also extremely important in determining optimal harvest rates \citep{hilborn1992quantitative}.  In general, the age-at-entry to the fishery relative to the age-at-maturity defines the optimal fishing mortality rate; as the age-at-entry to the fishery increases relative to the age-at-maturity, the higher the value of F$_{\rm{MSY}}$.


The current harvest strategy for Pacific halibut also includes the use of a minimum size limit in the directed commercial longline fishery. Halibut that are below the minimum legal size of 81.3 cm total length are required to be released and the IPHC assumes a 16\% mortality rate for released fish based on previous tagging studies.  Changes in the size limit also have implications for the F$_{\rm{MSY}}$ calculations, especially if release mortality rates are significant. In general, if halibut less than the minimum size limit are routinely captured, then reducing the minimum size limit will also result in reductions in the optimal harvest rate.  Note that the EBio calculation is based on all halibut greater than 66 cm (or 26 inches) so the definition of Ebio does not change with changes in size limits. 

Previous studies have examined the impacts of alternative size limits on the yield per recruit and spawning biomass per recruit using steady-state or equilibrium models \citep{clark1995re}.  Previous halibut studies have not considered the cumulative impacts of size-selective fishing on optimal harvest rate calculations for Pacific halibut.  Using a minimum size limit, faster growing individuals from a given cohort would experience higher total mortality rates over their lifetime as they recruit to the legal size at a younger age in comparison to slower growing individuals. The cumulative effect of fishing mortality imposes a higher total mortality rate on faster growing halibut; remaining individuals left in the population consist of slower growing individuals \citep{Taylor2005}.  The cumulative effects of size-selective fishing can, therefore, give the appearance of declining mean weight-at-age in the population.

Further, previous studies have not addressed how different release mortality rates affect  yield per recruit and optimal harvest rates for Pacific halibut.  \cite{coggins2007ecm} and \cite{pineiii2008car} demonstrated that for a given size limit, increases in discard mortality can lower the realized yield per recruit. This is also the case if both lower and upper size limits (i.e., slot-limit) are in place, where the assumption is that an upper size limit would add additional protection to large sexually mature females.

Maximizing  yield per recruit, or maximizing the total landed biomass is not always desirable from an economic perspective.  In the cases where there is a differential price structure for the size of fish landed (e.g., large fish fetch a higher price per pound  than small fish), it may be more desirable to fish at rates lower than $\rm{F_{msy}}$ in order to maximize value of all fish landed.  Moreover, fishing mortality rates associated with Maximum Economic Yield (MEY) are generally lower than fishing mortality rates associated with MSY \citep{gordon1954economic}.

In this paper, I develop an equilibrium model for Pacific halibut to examine how changes in biological components and size-specific fishing mortality impact estimates of optimum harvest rate calculations.  In addition to the directed fishery, the impacts of other constant removals from non-directed fisheries (e.g., bycatch) on estimates of optimal fishing mortality rates in the directed fishery are also examined.  The cumulative effects of size selective fishing on changes in mean weight-at-age are also explored. Size-at-age data from the 2011 setline survey are used to estimate growth curves for each regulatory area, and these growth curves are used to illustrate how relative differences in growth affect estimates of optimal harvest rates in each regulatory area.





% section introduction (end)
