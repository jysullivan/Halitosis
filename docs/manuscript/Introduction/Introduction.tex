%!TEX root = /Users/stevenmartell1/Documents/IPHC/SizeLimitz/docs/manuscript/SizeLimitz.tex
\section{Notes} % (fold)
\label{sec:notes}

\begin{itemize}
	\item Run growth model and calculate F0.1  using only parametric uncertainty in growth.
	\item Calculate F0.1 using parametric uncertainty in growth and a prior distribution for natural mortality.
\end{itemize}


% section notes (end)

\section*{Introduction} % (fold)
\label{sec:introduction}

The current harvest strategy for Pacific halibut includes the use of minimum size limits along with a constant exploitation rate policy to determine annual total catches for commercial and non-commercial fisheries.  Previous studies have examined the impacts of alternative size limits on the yield per recruit and spawning biomass per recruit using steady-state or equilibrium models \citep{clark1995re}, and more recently using a simulation modelling approach \citep{clark2006assessment}.  The current harvest policy for Pacific halibut utilizes a minimum legal size limit of 81.3cm (or 32 inches) total length.  Previous studies have not considered the impacts of size-selective fishing and release mortality rates on optimal harvest policies for Pacific halibut.  Using a minimum size limit, faster growing individuals from a given cohort would experience higher total mortality rates over their lifetime as they recruit to the legal size at a younger age in comparison to slower growing individuals. The cumulative effect over time imposes a higher total mortality rate on faster growing halibut and surviving individuals in the population are biased towards slower growing individuals \citep{Taylor2005}. The cumulative effects of size-selective fishing can give the appearance of declining mean weight-at-age in the population.

Previous yield per recruit and size limit interactions for Pacific halibut did not consider the effects of discard mortality rates. The underlying assumption in this case is that all fish captured and released will survive. \cite{coggins2007ecm} and \cite{pineiii2008car} demonstrated that for a given size limit, increases discard mortality can lower the realized yield per recruit. Moreover, non-zero discard mortality rates would reduce the potential benefit, if any, of an upper size limit as it would afford less opportunity to survive to the upper size limit, and reduce survivorship of captured individuals that not of legal size.

Maximizing the yield per recruit, or maximizing the total equilibrium yield is not always desirable from an economic perspective.  In the case of Pacific halibut, there is a differential price structure for the size of fish landed, where larger fish command much higher per pound prices than smaller fish. In such case, it may be more desirable to fish at rates lower than $\rm{F_{msy}}$, or F$_{0.1}$ in order to maximize the economic value.

In this paper, I develop an equilibrium model that address both the cumulative effects of size-selective fishing and discard mortality rates for Pacific halibut.  Yield per recruit, spawning biomass per recruit, equilibrium yield, depletion and landed value per recruit are compared using five alternative harvest policy scenarios: (1) the current size size limit of 81.3 cm (or 32 inches), (2) no size limits, (3) a 60 cm (or 26 inch) size limit, and (4) a slot limit with lower and upper bounds of 81.3--150 cm, and (5) a slot limit with lower and upper bounds of 60--150 cm.  Each of these five harvest policy scenarios are examined under alternative hypotheses about changes in size-at-age and changes in size-selectivity in the commercial fishery.




% section introduction (end)
