%!TEX root = /Users/stevenmartell1/Documents/IPHC/SizeLimitz/docs/manuscript/SizeLimitz.tex
\section*{Results} % (fold)
\label{sec:results}

\subsection*{Growth} % (fold)
\label{sub:growth}

Size-at-age data in each of the regulatory areas has very marked differences in both the mean length-at-age, and the distribution of age-classes (Figure \ref{fig:FIGS_fig:lengthAgeFitbySex}). Male and female halibut in area 2A are fast growing but tend to have a much smaller asymptotic size than halibut sampled in other regulatory areas.  Moreover, the age-composition in area 2A is truncated relative to other regions, with very few fish older than 17 years.  The coefficient of variation in length-at-age is much higher in areas 2B and 2C, especially for females.  Estimated growth rates for female halibut in these two areas is nearly linear for younger ages, and on average older halibut in these regions are much larger in comparison to other regions with old female halibut.  Area 4B is also another anomaly in that the age-distribution is much older, especially for males, with many sampled individuals beyond age 20.  Additional details about the growth model and estimated model parameters are found in Appendix \ref{sec:estimation_of_growth}.

\begin{figure}[htbp]
	\centering
		\includegraphics[width=\textwidth]{../../FIGS/fig:lengthAgeFitbySex.pdf}
	\caption{Length-at-age data by sex and regulatory area collected in the 2011 setline survey and corresponding estimated growth curves for each regulatory area.}
	\label{fig:FIGS_fig:lengthAgeFitbySex}
\end{figure}
% subsection growth (end)

\subsection*{Area-specific estimates of F$_{\rm{MSY}}$} % (fold)
\label{sub:area_specific_estimates_of_FMSY}

The relative equilibrium yield versus fishing mortality rate in each of the regulatory areas  (Figure \ref{fig:FIGS_fig:YeBase}) demonstrate the relative differences in expected yield based only on differences in halibut growth (size-at-age) in each of the regulatory areas.  For each of the regulatory areas shown in Figure \ref{fig:FIGS_fig:YeBase}, a minimum size limit of 81.3cm exists, steepness is fixed at an arbitrary value of 0.75, size-selectivity is the same in each area, and natural mortality rates are the same ($M=0.15$) for all areas.  The only biological difference between regulatory areas is the growth rate.  For each recruiting halibut in a specific regulatory area, the maximum yield per recruit would be obtained in area 2A.  Halibut in area 2A are very large at younger ages and more vulnerable to the fishing gear (selectivity) at a time when they are numerically more abundant.

In contrast, in Area 4C halibut obtain very large sizes, but the growth rate is much slower in comparison to 2A so fewer individuals are available to be harvested and less yield per recruit is obtained in this area.  The net result of this difference in growth rates is that estimates of F$_{\rm{MSY}}$ are lower in Area 4C relative to Area 2A.


\begin{figure}[htbp]
	\centering
		\includegraphics[height=4in]{../../FIGS/fig:YeBase.pdf}
	\caption{Relative equilibrium yield versus fishing mortality in the directed fishery assuming steepness $h=0.75$, a minimum size limit of 81.3cm, and no other sources of additional mortality. Vertical arrows pointing to the axis indicate the estimated F$_{\rm{MSY}}$ for each regulatory area.}
	\label{fig:FIGS_fig:YeBase}
\end{figure}

Equilibrium yield curves for each regulatory area under each of the 9 alternative scenarios are shown in Figure \ref{fig:FIGS_fig:YeALL} and the corresponding estimates of optimal exploitation rates ($1-exp(-F_{\rm{MSY}})$) are summarized in Table \ref{table:Umsy}.  Relative to the current harvest rate policy of 21.5\% and 16.125\%, estimates of optimal exploitation rates for Areas 2B and 2C are below the current 21.5\% value.   However, recall that this is based on the assumption of a Beverton-Holt stock recruitment relationship with a steepness value set at an arbitrary level of 0.75.  This arbitrary value was selected because it was consistent with the previous harvest rate policy, which was developed using two average density-independent recruitment values based on previous closed area assessment models.  The utility of S1 is to serve as a baseline in which to compare impacts of alternative size limits, and model assumptions,  on the estimates of optimal exploitation rates that would maximize the average long-term yield in each of the statistical areas.


\begin{figure}[htbp]
	\centering
		\includegraphics[width=\textwidth]{../../FIGS/fig:YeALL.pdf}
	\caption{Relative equilibrium yield versus fishing mortality for all 9 scenarios described in Table \ref{table:Scenarios}, where the different line colors denotes Regulatory area.}
	\label{fig:FIGS_fig:YeALL}
\end{figure}

\begin{table}
	\caption{Estimates of optimal exploitation rates for each regulatory area and scenario combination. Scenario descriptions are found on page \pageref{sub:scenarios}.}
	\label{table:Umsy}
	\begin{center}
		\begin{tabular}{c|ccccccccc}
		\hline
		Scenario & 2A & 2B & 2C & 3A & 3B & 4A & 4B & 4C & 4D\\
		\hline
		S1& 0.248 &0.197 &0.180 &0.297 &0.318 &0.260 &0.180 &0.176 &0.300\\
		S2& 0.248 &0.221 &0.221 &0.300 &0.321 &0.271 &0.225 &0.237 &0.304\\
		S3& 0.221 &0.180 &0.163 &0.256 &0.275 &0.229 &0.168 &0.163 &0.260\\
		S4& 0.201 &0.163 &0.151 &0.229 &0.245 &0.209 &0.155 &0.151 &0.233\\
		S5& 0.168 &0.137 &0.133 &0.172 &0.185 &0.168 &0.137 &0.133 &0.176\\
		S6& 0.180 &0.133 &0.120 &0.193 &0.205 &0.163 &0.124 &0.111 &0.193\\
		S7& 0.213 &0.176 &0.168 &0.252 &0.275 &0.233 &0.172 &0.172 &0.256\\
		S8& 0.260 &0.209 &0.197 &0.289 &0.311 &0.264 &0.197 &0.189 &0.293\\
		S9& 0.155 &0.129 &0.120 &0.180 &0.193 &0.163 &0.124 &0.120 &0.180\\
		\hline
		\end{tabular}
	\end{center}
\end{table}

%Imposing a maximum size limit.
The maximum size limit scenario results in slight increases in the estimates of $F_{\rm{MSY}}$ in areas where halibut grow to a sufficiently large size to benefit from such protection (Table \ref{table:Umsy}).  Imposing a maximum size limit does not result in any yield benefits in any of the regulatory areas (Scenario S2, Table \ref{table:MSY}).  In fact, in Areas 2A and 3B, there is a very small probability that an individual halibut would survive and grow to surpass the upper legal size limit of 140cm. There is only a minor improvement in the relative spawning biomass in areas 2A and 3B associated with a maximum size limit of 140cm (Table \ref{table:Bmsy}).  Whereas, there is a further reduction in the spawning biomass depletion in areas where halibut grow beyond the 140cm maximum size limit and fishing at $F_{\rm{MSY}}$, and the amount of spawning biomass reduction is proportional to the discard mortality rates.


%Removing size limits.
If size-limits were removed altogether, and there is no change in the size-selectivity of the commercial fishery, then estimates of $F_{\rm{MSY}}$ would have to be reduced (S3, Table \ref{table:Umsy}) in order to compensate for the increased total mortality rate associated with retaining fish smaller than 81.3cm.  Relative increases in overall yield do occur with the removal of the minimum size limits, as the yield per recruit in each area increases, with the exception of Area 4B.  However, this modest increase in overall yield does come at the expense of reducing spawning stock biomass, as well as, reducing the average size of landed fish.

Scenario 4 represents a shift in the commercial selectivity towards smaller fish, and the net impact of this shift is a reduction in the $F_{\rm{MSY}}$ values for each regulatory area, as well as, decreases in the overall landed yield (Tables \ref{table:Umsy} and \ref{table:MSY}).  Recall that this scenario was run with the current minimum size limit of 81.3cm in place and serves to show that minimum size-limits alone does not afford protection of spawning stock biomass if discard mortality rates are greater than 0.  Although corresponding increases in spawning biomass are observed in Table \ref{table:Bmsy} for scenario 4, this increase owes to the reduction in $F_{\rm{MSY}}$ that would be required to maximize yield if selectivity were to shift towards smaller fish.

\begin{table}
	\caption{Relative change in yield by regulatory area in comparison to S1 (status quo) for each of the alternative scenarios while fishing at rates defined in Table \ref{table:Umsy} (MSY based fishing mortality).}
	\label{table:MSY}
	\begin{center}
		\begin{tabular}{c|ccccccccc}
		\hline
		Scenario & 2A & 2B & 2C & 3A & 3B & 4A & 4B & 4C & 4D\\
		\hline
		S1&  0.00&  0.00&  0.00&  0.00&  0.00&  0.00&  0.00&  0.00&  0.00 \\
		S2&  0.00& -0.17& -0.27& -0.01&  0.00& -0.04& -0.33& -0.41& -0.01 \\
		S3&  0.65&  0.30&  0.11&  0.85&  0.78&  0.28& -0.01&  0.04&  0.45 \\
		S4& -0.04& -0.03& -0.02& -0.14& -0.15& -0.09& -0.04& -0.04& -0.14 \\
		S5& -1.41& -0.84& -0.69& -1.41& -1.44& -1.12& -0.71& -0.61& -1.47 \\
		S6& -0.72& -0.83& -1.00& -0.86& -0.89& -0.98& -1.00& -1.00& -0.91 \\
		S7&  0.32&  0.38&  0.48&  0.24&  0.24&  0.40&  0.45&  0.44&  0.26 \\
		S8&  1.10&  0.72&  0.66&  0.57&  0.55&  0.63&  0.69&  0.49&  0.60 \\
		S9& -1.13& -0.76& -0.69& -0.87& -0.88& -0.82& -0.78& -0.60& -0.92 \\
		%
	   %S1& 7.55 &5.41 &4.99 &5.72 &5.90 &5.81 &5.83 &4.61 &6.35\\
	   %S2& 7.54 &5.25 &4.72 &5.71 &5.89 &5.76 &5.49 &4.20 &6.34\\
	   %S3& 8.20 &5.72 &5.10 &6.58 &6.68 &6.09 &5.82 &4.65 &6.80\\
	   %S4& 7.50 &5.39 &4.97 &5.58 &5.74 &5.72 &5.79 &4.56 &6.21\\
	   %S5& 6.14 &4.57 &4.29 &4.31 &4.46 &4.68 &5.12 &3.99 &4.88\\
	   %S6& 6.83 &4.59 &3.99 &4.86 &5.01 &4.82 &4.83 &3.61 &5.44\\
	   %S7& 7.87 &5.79 &5.47 &5.96 &6.13 &6.20 &6.28 &5.05 &6.61\\
	   %S8& 8.65 &6.13 &5.64 &6.29 &6.44 &6.43 &6.52 &5.10 &6.95\\
	   %S9& 6.42 &4.66 &4.30 &4.85 &5.01 &4.98 &5.05 &4.00 &5.43\\
		\hline
		\end{tabular}
	\end{center}
\end{table}

The effects of other discard mortality, from non-directed fisheries, also plays a role in harvest policy calculations.  In scenario 5, a constant total harvest (i.e., bycatch from a trawl fishery) was imposed as an increasing fishing mortality rate with increasing directed $F_e$.  In this scenario, a constant level of bycatch results in a dramatic reduction in overall yield in the directed fishery (Table \ref{table:MSY}) and a reduction in the optimal harvest rate ($F_{\rm{MSY}}$) that would produce the maximum sustainable yield in the directed fishery.  This is similar to discarding undersized fish, but in this case the mortality rate imposed by the bycatch fishery increases with declining stock size.  Whereas, in the directed fishery, annual catches would scale down with reductions in exploitable biomass, and fishing mortality rates would scale down if the spawning biomass falls below B$_{30\%}$.  It should also be noted that the opposite result would occur where $F_{\rm{MSY}}$ estimates would increase if efforts were made to reduce bycatch in non-directed fisheries.  

%Bmsy table
\begin{table}
	\caption{Relative spawning biomass while fishing at $F_{\rm{MSY}}$ for each scenario and regulatory area.}
	\label{table:Bmsy}
	\begin{center}
		\begin{tabular}{c|ccccccccc}
		\hline
		Scenario & 2A & 2B & 2C & 3A & 3B & 4A & 4B & 4C & 4D\\
		\hline		%
		S1&$23.8$&$25.5$&$25.8$&$26.0$&$26.8$&$26.9$&$27.6$&$28.3$&$27.1$\tabularnewline
		S2&$23.9$&$24.4$&$24.2$&$25.7$&$26.6$&$26.2$&$25.5$&$25.4$&$26.9$\tabularnewline
		S3&$22.6$&$24.5$&$25.4$&$23.0$&$23.5$&$25.0$&$27.0$&$27.6$&$24.4$\tabularnewline
		S4&$24.8$&$26.1$&$26.7$&$27.5$&$28.3$&$27.7$&$27.8$&$28.8$&$28.5$\tabularnewline
		S5&$26.0$&$27.7$&$26.8$&$29.4$&$30.1$&$29.0$&$28.4$&$29.2$&$30.4$\tabularnewline
		S6&$24.1$&$25.9$&$25.7$&$26.0$&$26.6$&$27.2$&$26.7$&$28.2$&$27.3$\tabularnewline
		S7&$25.7$&$27.7$&$28.0$&$28.7$&$29.3$&$29.0$&$28.7$&$30.0$&$29.7$\tabularnewline
		S8&$20.5$&$22.4$&$22.4$&$24.3$&$25.1$&$24.7$&$24.1$&$25.6$&$25.6$\tabularnewline
		S9&$29.2$&$29.5$&$29.7$&$30.4$&$31.3$&$30.8$&$30.7$&$31.8$&$31.9$\tabularnewline
		\hline
		\end{tabular}
	\end{center}
\end{table}
% subsection area_specific_estimates_of_f__{msy_ (end)

\subsection*{Sensitivity to assumed parameter values} % (fold)
\label{sub:sensitivity_to_assumed_parameter_values}
Scenarios 6 and 7 examine how sensitive MSY-based reference points are to estimates of natural mortality rates.  The current assessment model assumes natural mortality is independent of size/age.  These two scenarios examine a size-effect in natural mortality.  In general, if $M$ is size/age independent, the increasing $M$ results in increases in the estimates of $F_{\rm{MSY}}$, and vice versa \citep{WalMart2004}.  Also increases in $M$ results in a decrease in MSY and the spawning biomass at $F_{\rm{MSY}}$.  If $M$ is size-dependent and decreases with size, estimates of $F_{\rm{MSY}}$  also decrease, and vice versa.  Natural mortality also plays a role in the general scaling of MSY; as fewer older fish are available for harvest due to high natural mortality rates, then the value of MSY decreases.  Hence mis-specification of $M$ can lead to biased estimates of other reference points (e.g., SB$_{100}$).

Lastly, estimates of $F_{\rm{MSY}}$ are very sensitive to the steepness of the stock-recruitment relationship.  With increasing steepness the corresponding estimates of $F_{\rm{MSY}}$ also increase, and vice versa.  The resilience of the stock to over-fishing decreases with decreasing values of steepness,  the overall yield declines and there are fewer recruits per unit of spawning biomass (lower stock productivity).

% subsection sensitivity_to_assumed_parameter_values (end)

\subsection*{Wastage in the directed fishery} % (fold)
\label{sub:wastage_in_the_directed_fishery}

In all scenarios where minimum or maximum size limits exist, wastage in the directed fishery increases with increasing fishing mortality.  Under the current status quo scenario (S1), the long-term average wastage in the directed fishery is estimated to be less than 5\% of the total landed yield in each of the regulatory areas if fishing mortality rates are  less that 0.25 (Figure \ref{fig:FIGS_fig:PercentWastage}). Note that for the purposes of this paper, as well as for the wastage estimates that go into the stock assessment model, it is assumed that the selectivity of the commercial fishery is the same as the estimated selectivity curve in the setline survey.

The use of an upper size limit (S2), results in increased wastage over the status quo scenario, but only in areas where halibut attain sufficiently large sizes. At low equilibrium fishing mortality rates wastage in Areas 2B, 2C, 4B and 4C are greater than 10\% of the landed catch due to large halibut (greater than 140cm) in these regions (Figure \ref{fig:FIGS_fig:PercentWastage}).  As fishing mortality rates increase and erode the size structure, wastage rates decline and effectively become the same as that of a minimum size limit only.  

\begin{figure}[htbp]
	\centering
		\includegraphics[width=\textwidth]{../../FIGS/fig:PercentWastage.pdf}
	\caption{Percent wastage in the directed fishery versus long-term equilibrium fishing mortality rate for each regulatory area. Scenario 1 consists of a minimum size limit of 81.3cm, scenario 2 includes an additional maximum size limit of 140cm, and scenario 4 consists of a 81.3cm minimum size limit and commercial selectivity shifting 10cm towards smaller fish.}
	\label{fig:FIGS_fig:PercentWastage}
\end{figure}

In scenario 4, where the commercial selectivity curve was shifted by 10cm towards smaller fish, the long-term average wastage in the directed fishery more that doubles what is currently assumed under the status quo scenario (Figure \ref{fig:FIGS_fig:PercentWastage}). Note that a minimum size limit of 81.3cm is also maintained in the S4 calculations.  Lastly, I also note here that under Scenario 3 (not shown in Figure \ref{fig:FIGS_fig:PercentWastage}), there is no wastage, as it is assumed that all fish harvested are landed.

% subsection wastage_in_the_directed_fishery (end)

\subsection*{Changes in mean weight-at-age} % (fold)
\label{sub:changes_in_mean_weight_at_age}

Mean size-at-age is predicted to decline with increasing size-selective fishing mortality, especially for age-classes that are fully recruited to the fishing gear (Figure \ref{fig:FIGS_fig:wbar_female}).  Ages less than 8 years are not expected to show much of a change in the mean weight-at-age because they are only partially recruited to the gear, and have not been subjected to intense fishing mortality.
\begin{figure}[htbp]
	\centering
		\includegraphics[width=\textwidth]{../../FIGS/fig:wbar_female.pdf}
	\caption{Expected change in mean weight-at-age for female halibut by regulatory area for ages 5, 10, 15, and 20 years.}
	\label{fig:FIGS_fig:wbar_female}
\end{figure}

There are substantial differences in how the predicted mean weight-at-age would change with increasing fishing mortality across regulatory areas.  This is a result of differences in growth rates among regulatory areas.  Despite these differences, the general pattern of cumulative size-selective fishing results larger changes in mean size for older individuals and little to no change for age classes that are not subject to intensive fishing.  Similar patterns are also evident in the raw size-at-age data collected from the setline survey (Figure \ref{fig:FIGS_fig:SAA_age6_14}).  In recent years, exploitation rates in area 2B are estimated to be greater than the target rate of 21.5\%.  Mean weight-at-age for age-6 fish in area 2B have varied little between 2002 and 2006; whereas there has been a decline in mean weight-at-age for ages 10 and 14 between 2007 and 2009 (Figure \ref{fig:FIGS_fig:SAA_age6_14}).  There is considerable variability in the observed size-at-age data from the set line survey in all of the regulatory areas.

\begin{figure}[htbp]
	\centering
		\includegraphics[width=\textwidth]{../../FIGS/fig:SAA_age6_14.pdf}
	\caption{Observed distribution of weight at ages 6,12 and 14 years for sampled female Pacific halibut in the setline survey between 1998 to 2011.}
	\label{fig:FIGS_fig:SAA_age6_14}
\end{figure}

% subsection changes_in_mean_weight_at_age (end)




% section results (end)

