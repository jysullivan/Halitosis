%!TEX root = /Users/stevenmartell1/Documents/IPHC/SizeLimitz/docs/manuscript/SizeLimitz.tex


\section*{Discussion} % (fold)
\label{sec:discussion}

%% Summary
Factors that affect estimates of optimal harvest rates come in two general forms: (1) biological components that define the underlying productivity of the stock, and (2) fishery components that affect size-at-entry and size-specific mortality. The former cannot be directly managed but must be taken into consideration in harevst policy, especially if there are temporal changes in stock productivity.  The latter can be directly controlled through a variety of tools that limit gear specifications, size limits, or even areas fished, to control size-at-entry into the fishery and reduce post release mortality rates.  In the case of Pacific halibut, there have been recent changes in size-at-age and there is also considerable variation in the size-at-age among the regulatory areas. Optimal harvest rate calculations for each of the regulatory areas will have to be updated frequently owing to continued changes, and differences, in size-at-age among regulatory areas.  Also, any changes in fisheries operations or changes in fishery regulations (e.g., change in size limits or bycatch) will also affect optimal harvest calculations.  In general, the underlying parameters that define the harvest policy should be updated on a routine, or even annual, basis to ensure that biological and technological changes are taken into account in updating harvest policy.

In this paper, we examined how changes in stock productivity and size-selectivity interact with estimates of fishing mortality rates that maximize long-term sustainable yields.  This was approximated using an age- and sex-structured equilibrium model conditioned on regulatory area size-at-age data from the 2011 setline survey.  Previous harvest policy development was based on a stochastic simulation model for Pacific halibut that is now dated because of: (a) continued changes in size-at-age for Pacific halibut, (b) a transition from area-based assessments to a coast-wide model, (c) recent recognition that fishery and survey selectivity must vary over time in the coast-wide assessment due to changes in the distribution of the stock, and (d) reduction in the bycatch levels in non-directed fisheries.  There have been previous equilibrium-based models for the development of harvest policy for Pacific halibut \citep[e.g.,][]{clark1995re}, but these models did not explicitly consider the effects of wastage and bycatch on optimal harvest rates.  Moreover, previous analyses were also limited to the core halibut areas (2B, 2C and 3A).  


At this point in time, this analysis should be considered a work in progress for the sole reason that a key population parameter (i.e., steepness) that defines the underlying stock productivity is not yet available for the new coast-wide assessment.  The steepness parameter was arbitrarily set at a value of 0.75 and was chosen because it resulted in estimates of $F_{\rm{MSY}}$ that are similar to those obtained by \cite{clark2006assessment}.  Nevertheless, the relative changes in $F_{\rm{MSY}}$ among regulatory areas based on differences in growth rates would not differ if a reliable estimate of steepness was available.  Also critical to optimal harvest rate calculations is the area-based selectivity of fishing gears that harvest Pacific halibut.  At present, the coast-wide assessment model assumes size-based selectivity for each fishing gear does not vary by regulatory area.  Next to differences in size-at-age, regulatory area differences in selectivity-at-age relative to maturity-at-age will also have a large impact on estimates of optimal exploitation rates.  The previous closed-area assessment models estimated marked differences in selectivity among regulatory areas 2B and 3A \citep{clark2006assessment}. The transition to a coast-wide model introduced a new assumption that size-based selectivity is the same for all regulatory areas.

 One issue, that has not been examined here, and has very important harvest policy implications is the initial recruitment and movement of halibut among regulatory areas. Area-specific optimal harvest rates are sensitive to movement of halibut among regulatory areas.  There has been a considerable effort in this regard to understand the movement of halibut \citep[e.g.,][]{loher2006seasonal,webster2009analysis} and what the potential implications are for harvest policy \citep{valero2009exploring,valero2010effect}.  In general, areas with a net migration loss are nearly equivalent to having a higher natural mortality rate in a closed area model.  The harvest policy implications in such a case would be to harvest at a higher rate, but the total removals would would scale down.  The opposite is true for an area that has a net migration increase, harvest at a lower rate, but the scale of the harvest increases due to immigration to the area.  If however, the objective is to maximize the yield from all areas combined, then the optimal harvest rate calculations are much more complex involving dispersal kernels for new recruits and migration transition matrices.  Under such circumstances it is not possible to make generalized statements about how optimal harvest rates would change because the answer depends on the relative migration coefficients among the regulatory areas.  \cite{valero2010effect} had made progress in this area and their early conclusions suggested that harvest policies to the north of area 2 would have fairly severe implications for area 2 itself due to downstream migration of halibut into this area.


It is intuitive to think that imposing a maximum size limit would afford protection to sexually mature fish that grow beyond the size limit and that this would lead to an increase in spawning biomass (or reduce the level of depletion).  This does occur, but only if there is a very low discard mortality rate associated with releasing fish.  In the case examined here, with 140cm size limit and fishing at $F_{\rm{MSY}}$, the spawning biomass in each regulatory area remains nearly the same or  declines in comparison to no maximum size limit.  The reason for this decline is related to a discard mortality rate of 0.16 per year, and under MSY-based harvest policies, values of $F_{\rm{MSY}}$ would increase in areas where halibut grow to sufficient size and a maximum size limit is used in the harvest policy.

Shifts in the directed commercial selectivity schedule towards smaller halibut pose a conservation concern if the discard mortality rate is greater than 0, even if minimum size limits are in place. If individual IFQ holders are not accountable for their discard mortality of undersized fish, then fishing can continue until their quota is filled with legal-sized halibut.  If there is a shift towards catching smaller fish, or the probability of capturing a legal-size fish in a given area is low, then the corresponding increase in discard mortality  results in a higher overall total mortality rate that may not be accounted for if the shift in selectivity goes undetected.  This is of particular concern in the current assessment of Pacific halibut, where the wastage calculation assumes the commercial fishery selectivity is the same as the top 33\% of the setline survey WPUE \citep{gilroy2009wastage}.  Moreover, estimated commercial selectivity is based on composition data from port samples, not on fish sampled on the boats at sea when the gear is being retrieved. In other words, the wastage calculation in the directed fishery is based on a tenuous assumption about how the commercial gear selects fish less than 81.3cm.

The harvest policy implications of undetected changes in selectivity and estimates of optimal exploitation rates are somewhat insensitive if the discard mortality rates are low or even negligible.  If under-sized fish are handled with extreme care such that survival rates are near 100\%, then the previous discussion about uncertainty in commercial selectivity for under-size fish is moot.  Moreover, the estimate of wastage would consist only of lost or abandoned gear.  However, if the release mortality rates are appreciable, then estimates of optimum harvest rates must also include release mortality associated with size-limits  \citep{goodyear1993spawning,coggins2007ecm}.  In general,  as the size limit increases the optimum fishing rate that would maximize yield increases exponentially.  This relationship is also the same for the fishing mortality rate that would deplete the spawning biomass to some target level.  The exponential increase in $F_{\rm{MSY}}$ occurs when individuals have had at least one chance to spawn before they become vulnerable to fishing.  \cite{pineiii2008car} demonstrated that as the release mortality rates increase, the potential of a minimum size limit to hedge against overfishing decreases as the release mortality rates increase.

If the observed changes in size-at-age are a result of cumulative size-selective fishing, then the largest changes in size-at-age would be expected in areas with higher fishing mortality rates.  In closed populations, the variance in size-at-age for older fish is expected to decrease with increasing fishing mortality rates.  Areas 2B and 2C are thought to have fairly high fishing mortality rates based on the results of the stock assessments and biomass apportionment \citep{Hare2012Rara}.  Based on the size-at-age data from the setline survey, the largest variance in size-at-age is found in Areas 2B and 2C, suggesting that these areas are heavily influence by migration into the area.


% concluding paragraph?
In the very near future, the Pacific halibut assessment model is likely to evolve to a more implicit spatial representation where estimated selectivities by regulatory area may differ (Ian Stewart, pers comm).  Given that steepness is unknown, and selectivity likely differs by regulatory area, it is not recommended to change the current harvest policy until, at a minimum, these two issues have been addressed.  Preferably, the full suite of biological factors, including dispersal and migration, and factors that affect selectivity would be included in the harvest policy analysis.

% section discussion (end)
